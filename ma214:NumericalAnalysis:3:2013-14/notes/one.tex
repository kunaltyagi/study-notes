\documentclass[12pt,letterpaper]{article}
\usepackage{amsmath,amsthm,amsfonts,amssymb,amscd}
\usepackage{fullpage}
\usepackage{lastpage}
\usepackage{enumerate}
\usepackage{fancyhdr}
\usepackage{mathrsfs}
\usepackage[margin=3cm]{geometry}
\setlength{\parindent}{0.0in}
\setlength{\parskip}{0.05in}
\title{Notes}

% Edit these as appropriate
\newcommand\course{MA 214}
\newcommand\semester{autumn 2014}  % <-- current semester
\newcommand\asgnname{Notes}         % <-- assignment name
\newcommand\yourname{Kunal Tyagi}  % <-- your name
\newcommand\login{kunaltyagi}          % <-- your CS login

\newenvironment{answer}[1]{
  \subsubsection*{%some prelude
  \asgnname.#1}
}{\newpage}

\pagestyle{fancyplain}
\headheight 35pt
\lhead{\yourname\ (\login)\\\course\ --- \semester}
\chead{\textbf{\Large MA \asgnname}}
\rhead{\today}
\headsep 10pt

\begin{document}

\begin{answer}{1}
    \begin{itemize}
        \item $ N = \pm (0.d_0 d_1 \ldots d_n)_\beta \beta^e$, here we have a number N with n significant digits in base $\beta$, with \textbf{exponent} as e and $(0.d_0 d_1 \ldots d_n)_\beta$ as \textbf{mantissa}
        \item If $x^*$ is approx. for x, then 
        \begin{itemize}
            \item $\mid x-x^* \mid$ is \textbf{Absolute Error} (AE)
            \item $\mid \frac{x-x^*}{x} \mid$ is the \textbf{Relative Error} (RE)
            \item RE can be approximated if x$\approx x^*$ by $\mid \frac{x-x^*}{x^*} \mid$ since $\frac{\alpha}{1-\alpha} \approx \alpha$ if $\alpha \ll$ 1
        \end{itemize}
        \item $x^*$ approximates x to t significant digits if RE is $\le 5 \times 10^{-t}$
        \item Loss of Significant Digits is caused by 
        \begin{itemize}
            \item addition/subtraction of quantities of nearly same value
            \item multiplication by a very large number
            \item division by a number very close to zero
        \end{itemize}
        \item To cure this, try to multiply with denominator/numerator's complement
        \item \textbf{Error Propagation}
        \begin{itemize}
            \item Condition/Sensitivity = max{$\mid \frac{f(x)-f(x^*)}{f(x)}\mid \div \mid \frac{x-x^*}{x} \mid$} for small RE. It is $\approx \mid \frac{f'(x)x}{f(x)}\mid$. The larger the number, the more the function is \textbf{ill-conditioned}
            \item Instability: Error of one step carries forward in next step and is magnified.
        \end{itemize}
        \item \textbf{Indermediate Value Theorem (IVT) for Continuous functions}: For all $f_{min}(x) \le F \le f_{max}(x)$, there exists at least one point c $\in$ [a,b]
        \item Use it in addition of $f(x_i)$ and in integration, where the sign of multiplicand of f(x) doesn't change \textbf{IMPORTANT}
        \item 
        
    \end{itemize}
\end{answer}

\begin{answer}{2}
Answer to problem 2 goes here, etc.
\end{answer}

\end{document}
