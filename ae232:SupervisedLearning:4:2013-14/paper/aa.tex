%                                                                 aa.dem
% AA vers. 8.2, LaTeX class for A&A
% demonstration file
%                                                       (c) EDP Sciences
%-----------------------------------------------------------------------
%
%\documentclass[referee]{aa} % for a referee version
%\documentclass[onecolumn]{aa} % for a paper on 1 column  
%\documentclass[longauth]{aa} % for the long lists of affiliations 
%\documentclass[rnote]{aa} % for the research notes
%\documentclass[letter]{aa} % for the letters 
%\documentclass[bibyear]{aa} % if the references are not structured 
% according to the author-year natbib style

%
\documentclass{aa}  

%
\usepackage{graphicx}
%%%%%%%%%%%%%%%%%%%%%%%%%%%%%%%%%%%%%%%%
\usepackage{txfonts}
%%%%%%%%%%%%%%%%%%%%%%%%%%%%%%%%%%%%%%%%
%\usepackage[options]{hyperref}
% To add links in your PDF file, use the package "hyperref"
% with options according to your LaTeX or PDFLaTeX drivers.
%
\begin{document} 


   \title{Energy Audit of Active Flow Control Systems}

   %\subtitle{I. Overviewing the $\kappa$-mechanism}

   \author{Prof. A M Pradeep
          \inst{1}
          \and
          Kunal Tyagi\inst{2}\fnmsep%\thanks{Just to show the usage of the elements in the author field}
          }

   \institute{Indian Institute of Technology Bombay,\\
              Projct Guide
              \email{ampradeep@aero.iitb.ac.in}
         \and
             Indian Institute of Technology, Bombay\\
             Sophomore
             \email{tyagi.kunal@outlook.com}
             %\thanks{The university of heaven temporarily does not accept e-mails}
             }

   \date{\today}

% \abstract{}{}{}{}{} 
% 5 {} token are mandatory
 
  \abstract
  % context heading (optional)
  % {} leave it empty if necessary  
   {}
  % aims heading (mandatory)
   {It is known that active flow control systems produce a lot of
   changes in the flow which many time completely change
   characterisitics of the flow making comparison between 
   different flow control systems very difficult. However, all the
   systems can be compared on the basis of their energy efficiency,
   i.e., the ratio of change in energy of the flow and the energy 
   comsumed by the system to bring the change.}
  % methods heading (mandatory)
   {All the control systems typically consume the most energy during
   the control loop. Another significant portion of the energy in
   many systems goes towards manipulating the flow. Despite the
   different methods of energy consumption by the flow control
   systems, the energy consumed can be calculated. On the other hand,
   to calculate the energy dissapated/gained by the flow, we have to
   check what the control mechanism is , and the compute the energy
   change of the flow}
  % results heading (mandatory)
   {RESULT}
  % conclusions heading (optional), leave it empty if necessary 
   {}

   \keywords{energy audit --
            active flow
            }

   \maketitle
%
%________________________________________________________________

\section{Introduction}

   The ability to manipulate a flow field to bring about a desired
   effect is of immense technological importance, as its potential
   benefits include savings in fuel costs, economically competitive
   and environmentally sound processes involving fluid flows.
   (\cite{GadElHak})

   Although passive flow control has been used aggressively in various
   fileds, active flow control has consistently remained in the back
   seat not due to ineffectiveness of the said method, but due to 
   lack of reliable comparison between the different mechanisms, which
   differ completely in their method of flow control, yet give
   comparable results.
   
   The motivation for his article is to investigate and categorise the
   different active flow methods and compare them using some set of
   relevant parameters.
   
   With this aim the local, active flow control methods are investigated
   on the basis of energy auditing.

   %The prospective of intustrial applications serve as a further 
   %motivation for this study.

%__________________________________________________________________

\section{Active Flow Control methods}

\begin{comment}
%                                     Two column figure (place early!)
%______________________________________________ Gamma_1 (lg rho, lg e)
   \begin{figure*}
   \centering
   %%%\includegraphics{empty.eps}
   %%%\includegraphics{empty.eps}
   %%%\includegraphics{empty.eps}
   \caption{Adiabatic exponent $\Gamma_1$.
               $\Gamma_1$ is plotted as a function of
               $\lg$ internal energy $\mathrm{[erg\,g^{-1}]}$ and $\lg$
               density $\mathrm{[g\,cm^{-3}]}$.}
              \label{FigGam}%
    \end{figure*}
%
\end{comment}
   
   In this section the methods used by various active flow control
   systems will bew reviewed. Their performance characteristics will
   be rewritten in terms of energy input and consumption so as to
   facilitate easier categorisation and comparison.

   Largely, the methods can be categorised by the following mechanisms:
   \begin{itemize}
      \item Cavity Oscillation control
      \item Transition Control
      \item Boundary Layer Seperation \& Bubble Generation Control
      \item "On Demand" Vortex Generators
      \item Zero Net Mass Flux (Synthetic Jet) Actuators
      \end{itemize}

\section{Assumptions}

   The calculations are done considering the following assumptions:
   \begin{enumerate}
	  \item Energy consumed by the mechanism excludes the energy required 
	  for feedback or feedforward loops, and includes the energy required
	  to maintain or change the state of the mechanism.
	  \item Energy introduced into the flow is calculated considering the
	  energy of vortices, flow, including energy losses due to turbulence,
	  viscous friction, or other disturbances like bubble formation.
	  \item The calculations disregard the change in energy due to heat
	  flow into or out of the mechanism and/or the flow.
   \end{enumerate}

\section{Symbols}

   The following sections use standard and known symbols for as many 
   quantities as possible, however, some symbols may differ depending on the
   context and requirement. An exhaustive list of the symbols used hereafter
   in Einstein's notation:
   \[
      \begin{array}{lp{0.8\linewidth}}
		 \~{A}			& Control surface						\\
		 a_{o}			& Speed of sound						\\
		 c				& span									\\
		 C_{D}			& Drag coefficient(\equiv2D/\rho _{\infty}U^{2}_{\infty}c; per unit span)\\
		 M				& Mach number of the unperturbed flow	\\
		 U				& Flow velocity							\\
		 X_{0}			& Initial conditions; also Conditions at x_{i} = 0\\
		 X_{o}			& Conditons outside boundary layer		\\
		 X_{u}			& Unperturbed flow conditions			\\
		 X_{p}			& Perturbed flow conditions				\\
		 X_{\infty}		& Far-field, or ambient conditions		\\
		 X_{w}			& Conditions at the wall				\\
		 X_{rms}		& Root mean Square value				\\
		 X_{in}			& 
      \end{array}
   \]

   
\begin{comment}
   For the one-zone-model Baker obtains necessary conditions
   for dynamical, secular and vibrational (or pulsational)
   stability (Eqs.\ (34a,\,b,\,c) in Baker \cite{baker}). Using Baker's
   notation:
   \[
      \begin{array}{lp{0.8\linewidth}}
         M_{r}  & mass internal to the radius $r$     \\
         m               & mass of the zone                    \\
         r_0             & unperturbed zone radius             \\
         \rho_0          & unperturbed density in the zone     \\
         T_0             & unperturbed temperature in the zone \\
         L_{r0}          & unperturbed luminosity              \\
         E_{\mathrm{th}} & thermal energy of the zone
      \end{array}
   \]
\noindent
   and with the definitions of the \emph{local cooling time\/}
   (see Fig.~\ref{FigGam})
   \begin{equation}
      \tau_{\mathrm{co}} = \frac{E_{\mathrm{th}}}{L_{r0}} \,,
   \end{equation}
   and the \emph{local free-fall time}
   \begin{equation}
      \tau_{\mathrm{ff}} =
         \sqrt{ \frac{3 \pi}{32 G} \frac{4\pi r_0^3}{3 M_{\mathrm{r}}}
}\,,
   \end{equation}
   Baker's $K$ and $\sigma_0$ have the following form:
   \begin{eqnarray}
      \sigma_0 & = & \frac{\pi}{\sqrt{8}}
                     \frac{1}{ \tau_{\mathrm{ff}}} \\
      K        & = & \frac{\sqrt{32}}{\pi} \frac{1}{\delta}
                        \frac{ \tau_{\mathrm{ff}} }
                             { \tau_{\mathrm{co}} }\,;
   \end{eqnarray}
   where $ E_{\mathrm{th}} \approx m (P_0/{\rho_0})$ has been used and
   \begin{equation}
   \begin{array}{l}
      \delta = - \left(
                    \frac{ \partial \ln \rho }{ \partial \ln T }
                 \right)_P \\
      e=mc^2
   \end{array}
   \end{equation}
   is a thermodynamical quantity which is of order $1$ and equal to $1$
   for nonreacting mixtures of classical perfect gases. The physical
   meaning of $ \sigma_0 $ and $K$ is clearly visible in the equations
   above. $\sigma_0$ represents a frequency of the order one per
   free-fall time. $K$ is proportional to the ratio of the free-fall
   time and the cooling time. Substituting into Baker's criteria, using
   thermodynamic identities and definitions of thermodynamic quantities,
   \begin{displaymath}
      \Gamma_1      = \left( \frac{ \partial \ln P}{ \partial\ln \rho}
                           \right)_{S}    \, , \;
      \chi^{}_\rho  = \left( \frac{ \partial \ln P}{ \partial\ln \rho}
                           \right)_{T}    \, , \;
      \kappa^{}_{P} = \left( \frac{ \partial \ln \kappa}{ \partial\ln P}
                           \right)_{T}
   \end{displaymath}
   \begin{displaymath}
      \nabla_{\mathrm{ad}} = \left( \frac{ \partial \ln T}
                             { \partial\ln P} \right)_{S} \, , \;
      \chi^{}_T       = \left( \frac{ \partial \ln P}
                             { \partial\ln T} \right)_{\rho} \, , \;
      \kappa^{}_{T}   = \left( \frac{ \partial \ln \kappa}
                             { \partial\ln T} \right)_{T}
   \end{displaymath}
   one obtains, after some pages of algebra, the conditions for
   \emph{stability\/} given
   below:
   \begin{eqnarray}
      \frac{\pi^2}{8} \frac{1}{\tau_{\mathrm{ff}}^2}
                ( 3 \Gamma_1 - 4 )
         & > & 0 \label{ZSDynSta} \\
      \frac{\pi^2}{\tau_{\mathrm{co}}
                   \tau_{\mathrm{ff}}^2}
                   \Gamma_1 \nabla_{\mathrm{ad}}
                   \left[ \frac{ 1- 3/4 \chi^{}_\rho }{ \chi^{}_T }
                          ( \kappa^{}_T - 4 )
                        + \kappa^{}_P + 1
                   \right]
        & > & 0 \label{ZSSecSta} \\
     \frac{\pi^2}{4} \frac{3}{\tau_{ \mathrm{co} }
                              \tau_{ \mathrm{ff} }^2
                             }
         \Gamma_1^2 \, \nabla_{\mathrm{ad}} \left[
                                   4 \nabla_{\mathrm{ad}}
                                   - ( \nabla_{\mathrm{ad}} \kappa^{}_T
                                     + \kappa^{}_P
                                     )
                                   - \frac{4}{3 \Gamma_1}
                                \right]
        & > & 0   \label{ZSVibSta}
   \end{eqnarray}
%
   For a physical discussion of the stability criteria see Baker
   (\cite{baker}) or Cox (\cite{cox}).

   We observe that these criteria for dynamical, secular and
   vibrational stability, respectively, can be factorized into
   \begin{enumerate}
      \item a factor containing local timescales only,
      \item a factor containing only constitutive relations and
         their derivatives.
   \end{enumerate}
   The first factors, depending on only timescales, are positive
   by definition. The signs of the left hand sides of the
   inequalities~(\ref{ZSDynSta}), (\ref{ZSSecSta}) and (\ref{ZSVibSta})
   therefore depend exclusively on the second factors containing
   the constitutive relations. Since they depend only
   on state variables, the stability criteria themselves are \emph{
   functions of the thermodynamic state in the local zone}. The
   one-zone stability can therefore be determined
   from a simple equation of state, given for example, as a function
   of density and
   temperature. Once the microphysics, i.e.\ the thermodynamics
   and opacities (see Table~\ref{KapSou}), are specified (in practice
   by specifying a chemical composition) the one-zone stability can
   be inferred if the thermodynamic state is specified.
   The zone -- or in
   other words the layer -- will be stable or unstable in
   whatever object it is imbedded as long as it satisfies the
   one-zone-model assumptions. Only the specific growth rates
   (depending upon the time scales) will be different for layers
   in different objects.

%__________________________________________________ One column table
   \begin{table}
      \caption[]{Opacity sources.}
         \label{KapSou}
     $$ 
         \begin{array}{p{0.5\linewidth}l}
            \hline
            \noalign{\smallskip}
            Source      &  T / {[\mathrm{K}]} \\
            \noalign{\smallskip}
            \hline
            \noalign{\smallskip}
            Yorke 1979, Yorke 1980a & \leq 1700^{\mathrm{a}}     \\
%           Yorke 1979, Yorke 1980a & \leq 1700             \\
            Kr\"ugel 1971           & 1700 \leq T \leq 5000 \\
            Cox \& Stewart 1969     & 5000 \leq             \\
            \noalign{\smallskip}
            \hline
         \end{array}
     $$ 
   \end{table}
%
   We will now write down the sign (and therefore stability)
   determining parts of the left-hand sides of the inequalities
   (\ref{ZSDynSta}), (\ref{ZSSecSta}) and (\ref{ZSVibSta}) and thereby
   obtain \emph{stability equations of state}.

   The sign determining part of inequality~(\ref{ZSDynSta}) is
   $3\Gamma_1 - 4$ and it reduces to the
   criterion for dynamical stability
   \begin{equation}
     \Gamma_1 > \frac{4}{3}\,\cdot
   \end{equation}
   Stability of the thermodynamical equilibrium demands
   \begin{equation}
      \chi^{}_\rho > 0, \;\;  c_v > 0\, ,
   \end{equation}
   and
   \begin{equation}
      \chi^{}_T > 0
   \end{equation}
   holds for a wide range of physical situations.
   With
   \begin{eqnarray}
      \Gamma_3 - 1 = \frac{P}{\rho T} \frac{\chi^{}_T}{c_v}&>&0\\
      \Gamma_1     = \chi_\rho^{} + \chi_T^{} (\Gamma_3 -1)&>&0\\
      \nabla_{\mathrm{ad}}  = \frac{\Gamma_3 - 1}{\Gamma_1}         &>&0
   \end{eqnarray}
   we find the sign determining terms in inequalities~(\ref{ZSSecSta})
   and (\ref{ZSVibSta}) respectively and obtain the following form
   of the criteria for dynamical, secular and vibrational
   \emph{stability}, respectively:
   \begin{eqnarray}
      3 \Gamma_1 - 4 =: S_{\mathrm{dyn}}      > & 0 & \label{DynSta}  \\
%
      \frac{ 1- 3/4 \chi^{}_\rho }{ \chi^{}_T } ( \kappa^{}_T - 4 )
         + \kappa^{}_P + 1 =: S_{\mathrm{sec}} > & 0 & \label{SecSta} \\
%
      4 \nabla_{\mathrm{ad}} - (\nabla_{\mathrm{ad}} \kappa^{}_T
                             + \kappa^{}_P)
                             - \frac{4}{3 \Gamma_1} =: S_{\mathrm{vib}}
                                      > & 0\,.& \label{VibSta}
   \end{eqnarray}
   The constitutive relations are to be evaluated for the
   unperturbed thermodynamic state (say $(\rho_0, T_0)$) of the zone.
   We see that the one-zone stability of the layer depends only on
   the constitutive relations $\Gamma_1$,
   $\nabla_{\mathrm{ad}}$, $\chi_T^{},\,\chi_\rho^{}$,
   $\kappa_P^{},\,\kappa_T^{}$.
   These depend only on the unperturbed
   thermodynamical state of the layer. Therefore the above relations
   define the one-zone-stability equations of state
   $S_{\mathrm{dyn}},\,S_{\mathrm{sec}}$
   and $S_{\mathrm{vib}}$. See Fig.~\ref{FigVibStab} for a picture of
   $S_{\mathrm{vib}}$. Regions of secular instability are
   listed in Table~1.

%
%                                                One column figure
%----------------------------------------------------------- S_vib
   \begin{figure}
   \centering
   %%%\includegraphics[width=3cm]{empty.eps}
      \caption{Vibrational stability equation of state
               $S_{\mathrm{vib}}(\lg e, \lg \rho)$.
               $>0$ means vibrational stability.
              }
         \label{FigVibStab}
   \end{figure}
%
%______________________________________________________________

\end{comment}

\section{Conclusions}

   \begin{enumerate}
      \item The conditions for the stability of static, radiative
         layers in gas spheres, as described by Baker's (\cite{baker})
         standard one-zone model, can be expressed as stability
         equations of state. These stability equations of state depend
         only on the local thermodynamic state of the layer.
      \item If the constitutive relations -- equations of state and
         Rosseland mean opacities -- are specified, the stability
         equations of state can be evaluated without specifying
         properties of the layer.
      \item For solar composition gas the $\kappa$-mechanism is
         working in the regions of the ice and dust features
         in the opacities, the $\mathrm{H}_2$ dissociation and the
         combined H, first He ionization zone, as
         indicated by vibrational instability. These regions
         of instability are much larger in extent and degree of
         instability than the second He ionization zone
         that drives the Cephe{\"\i}d pulsations.
   \end{enumerate}
   
%\begin{acknowledgements}
%      Part of this work was supported by the German
%      \emph{Deut\-sche For\-schungs\-ge\-mein\-schaft, DFG\/} project
%      number Ts~17/2--1.
%\end{acknowledgements}


%-------------------------------------------------------------------

\begin{thebibliography}{}

  \bibitem[1]{GadElHak} Gad-el-Hak, Mohammed 2000,
    in Flow Control
      
  \bibitem[2]{langley} L.G. Pack \& R.D. Joshlin,
   in Overview of Active Flow Control at NASA Langley Research Center
      
  \bibitem[2000]{baker} Gad-el-Hak, Mohammed 2000,
    in Flow Control

  \bibitem[1988]{balluch} Balluch, M. 1988,
    A\&A, 200, 58

  \bibitem[1980]{cox} Cox, J. P. 1980,
    Theory of Stellar Pulsation
    (Princeton University Press, Princeton) 165

  \bibitem[1969]{cox69} Cox, A. N.,\& Stewart, J. N. 1969,
    Academia Nauk, Scientific Information 15, 1

  \bibitem[1980]{mizuno} Mizuno H. 1980,
    Prog. Theor. Phys., 64, 544
   
  \bibitem[1987]{tscharnuter} Tscharnuter W. M. 1987,
    A\&A, 188, 55
  
  \bibitem[1992]{terlevich} Terlevich, R. 1992, in ASP Conf. Ser. 31, 
    Relationships between Active Galactic Nuclei and Starburst Galaxies, 
    ed. A. V. Filippenko, 13

  \bibitem[1980a]{yorke80a} Yorke, H. W. 1980a,
    A\&A, 86, 286

  \bibitem[1997]{zheng} Zheng, W., Davidsen, A. F., Tytler, D. \& Kriss, G. A.
    1997, preprint
\end{thebibliography}

\end{document}

%%%%%%%%%%%%%%%%%%%%%%%%%%%%%%%%%%%%%%%%%%%%%%%%%%%%%%%%%%%%%%
Examples for figures using graphicx
A guide "Using Imported Graphics in LaTeX2e"  (Keith Reckdahl)
is available on a lot of LaTeX public servers or ctan mirrors.
The file is : epslatex.pdf 
%%%%%%%%%%%%%%%%%%%%%%%%%%%%%%%%%%%%%%%%%%%%%%%%%%%%%%%%%%%%%%

%_____________________________________________________________
%                 A figure as large as the width of the column
%-------------------------------------------------------------
   \begin{figure}
   \centering
   \includegraphics[width=\hsize]{empty.eps}
      \caption{Vibrational stability equation of state
               $S_{\mathrm{vib}}(\lg e, \lg \rho)$.
               $>0$ means vibrational stability.
              }
         \label{FigVibStab}
   \end{figure}
%
%_____________________________________________________________
%                                    One column rotated figure
%-------------------------------------------------------------
   \begin{figure}
   \centering
   \includegraphics[angle=-90,width=3cm]{empty.eps}
      \caption{Vibrational stability equation of state
               $S_{\mathrm{vib}}(\lg e, \lg \rho)$.
               $>0$ means vibrational stability.
              }
         \label{FigVibStab}
   \end{figure}
%
%_____________________________________________________________
%                        Figure with caption on the right side 
%-------------------------------------------------------------
   \begin{figure}
   \sidecaption
   \includegraphics[width=3cm]{empty.eps}
      \caption{Vibrational stability equation of state
               $S_{\mathrm{vib}}(\lg e, \lg \rho)$.
               $>0$ means vibrational stability.
              }
         \label{FigVibStab}
   \end{figure}
%
%_____________________________________________________________
%
%_____________________________________________________________
%                                Figure with a new BoundingBox 
%-------------------------------------------------------------
   \begin{figure}
   \centering
   \includegraphics[bb=10 20 100 300,width=3cm,clip]{empty.eps}
      \caption{Vibrational stability equation of state
               $S_{\mathrm{vib}}(\lg e, \lg \rho)$.
               $>0$ means vibrational stability.
              }
         \label{FigVibStab}
   \end{figure}
%
%_____________________________________________________________
%
%_____________________________________________________________
%                                      The "resizebox" command 
%-------------------------------------------------------------
   \begin{figure}
   \resizebox{\hsize}{!}
            {\includegraphics[bb=10 20 100 300,clip]{empty.eps}
      \caption{Vibrational stability equation of state
               $S_{\mathrm{vib}}(\lg e, \lg \rho)$.
               $>0$ means vibrational stability.
              }
         \label{FigVibStab}
   \end{figure}
%
%______________________________________________________________
%
%_____________________________________________________________
%                                             Two column Figure 
%-------------------------------------------------------------
   \begin{figure*}
   \resizebox{\hsize}{!}
            {\includegraphics[bb=10 20 100 300,clip]{empty.eps}
      \caption{Vibrational stability equation of state
               $S_{\mathrm{vib}}(\lg e, \lg \rho)$.
               $>0$ means vibrational stability.
              }
         \label{FigVibStab}
   \end{figure*}
%
%______________________________________________________________
%
%_____________________________________________________________
%                                             Simple A&A Table
%_____________________________________________________________
%
\begin{table}
\caption{Nonlinear Model Results}             % title of Table
\label{table:1}      % is used to refer this table in the text
\centering                          % used for centering table
\begin{tabular}{c c c c}        % centered columns (4 columns)
\hline\hline                 % inserts double horizontal lines
HJD & $E$ & Method\#2 & Method\#3 \\    % table heading 
\hline                        % inserts single horizontal line
   1 & 50 & $-837$ & 970 \\      % inserting body of the table
   2 & 47 & 877    & 230 \\
   3 & 31 & 25     & 415 \\
   4 & 35 & 144    & 2356 \\
   5 & 45 & 300    & 556 \\ 
\hline                                   %inserts single line
\end{tabular}
\end{table}
%
%_____________________________________________________________
%                                             Two column Table 
%_____________________________________________________________
%
\begin{table*}
\caption{Nonlinear Model Results}             
\label{table:1}      
\centering          
\begin{tabular}{c c c c l l l }     % 7 columns 
\hline\hline       
                      % To combine 4 columns into a single one 
HJD & $E$ & Method\#2 & \multicolumn{4}{c}{Method\#3}\\ 
\hline                    
   1 & 50 & $-837$ & 970 & 65 & 67 & 78\\  
   2 & 47 & 877    & 230 & 567& 55 & 78\\
   3 & 31 & 25     & 415 & 567& 55 & 78\\
   4 & 35 & 144    & 2356& 567& 55 & 78 \\
   5 & 45 & 300    & 556 & 567& 55 & 78\\
\hline                  
\end{tabular}
\end{table*}
%
%-------------------------------------------------------------
%                                          Table with notes 
%-------------------------------------------------------------
%
% A single note
\begin{table}
\caption{\label{t7}Spectral types and photometry for stars in the
  region.}
\centering
\begin{tabular}{lccc}
\hline\hline
Star&Spectral type&RA(J2000)&Dec(J2000)\\
\hline
69           &B1\,V     &09 15 54.046 & $-$50 00 26.67\\
49           &B0.7\,V   &*09 15 54.570& $-$50 00 03.90\\
LS~1267~(86) &O8\,V     &09 15 52.787&11.07\\
24.6         &7.58      &1.37 &0.20\\
\hline
LS~1262      &B0\,V     &09 15 05.17&11.17\\
MO 2-119     &B0.5\,V   &09 15 33.7 &11.74\\
LS~1269      &O8.5\,V   &09 15 56.60&10.85\\
\hline
\end{tabular}
\tablefoot{The top panel shows likely members of Pismis~11. The second
panel contains likely members of Alicante~5. The bottom panel
displays stars outside the clusters.}
\end{table}
%
% More notes
%
\begin{table}
\caption{\label{t7}Spectral types and photometry for stars in the
  region.}
\centering
\begin{tabular}{lccc}
\hline\hline
Star&Spectral type&RA(J2000)&Dec(J2000)\\
\hline
69           &B1\,V     &09 15 54.046 & $-$50 00 26.67\\
49           &B0.7\,V   &*09 15 54.570& $-$50 00 03.90\\
LS~1267~(86) &O8\,V     &09 15 52.787&11.07\tablefootmark{a}\\
24.6         &7.58\tablefootmark{1}&1.37\tablefootmark{a}   &0.20\tablefootmark{a}\\
\hline
LS~1262      &B0\,V     &09 15 05.17&11.17\tablefootmark{b}\\
MO 2-119     &B0.5\,V   &09 15 33.7 &11.74\tablefootmark{c}\\
LS~1269      &O8.5\,V   &09 15 56.60&10.85\tablefootmark{d}\\
\hline
\end{tabular}
\tablefoot{The top panel shows likely members of Pismis~11. The second
panel contains likely members of Alicante~5. The bottom panel
displays stars outside the clusters.\\
\tablefoottext{a}{Photometry for MF13, LS~1267 and HD~80077 from
Dupont et al.}
\tablefoottext{b}{Photometry for LS~1262, LS~1269 from
Durand et al.}
\tablefoottext{c}{Photometry for MO2-119 from
Mathieu et al.}
}
\end{table}
%
%-------------------------------------------------------------
%                                       Table with references 
%-------------------------------------------------------------
%
\begin{table*}[h]
 \caption[]{\label{nearbylistaa2}List of nearby SNe used in this work.}
\begin{tabular}{lccc}
 \hline \hline
  SN name &
  Epoch &
 Bands &
  References \\
 &
  (with respect to $B$ maximum) &
 &
 \\ \hline
1981B   & 0 & {\it UBV} & 1\\
1986G   &  $-$3, $-$1, 0, 1, 2 & {\it BV}  & 2\\
1989B   & $-$5, $-$1, 0, 3, 5 & {\it UBVRI}  & 3, 4\\
1990N   & 2, 7 & {\it UBVRI}  & 5\\
1991M   & 3 & {\it VRI}  & 6\\
\hline
\noalign{\smallskip}
\multicolumn{4}{c}{ SNe 91bg-like} \\
\noalign{\smallskip}
\hline
1991bg   & 1, 2 & {\it BVRI}  & 7\\
1999by   & $-$5, $-$4, $-$3, 3, 4, 5 & {\it UBVRI}  & 8\\
\hline
\noalign{\smallskip}
\multicolumn{4}{c}{ SNe 91T-like} \\
\noalign{\smallskip}
\hline
1991T   & $-$3, 0 & {\it UBVRI}  &  9, 10\\
2000cx  & $-$3, $-$2, 0, 1, 5 & {\it UBVRI}  & 11\\ %
\hline
\end{tabular}
\tablebib{(1)~\citet{branch83};
(2) \citet{phillips87}; (3) \citet{barbon90}; (4) \citet{wells94};
(5) \citet{mazzali93}; (6) \citet{gomez98}; (7) \citet{kirshner93};
(8) \citet{patat96}; (9) \citet{salvo01}; (10) \citet{branch03};
(11) \citet{jha99}.
}
\end{table}
%_____________________________________________________________
%                      A rotated Two column Table in landscape  
%-------------------------------------------------------------
\begin{sidewaystable*}
\caption{Summary for ISOCAM sources with mid-IR excess 
(YSO candidates).}\label{YSOtable}
\centering
\begin{tabular}{crrlcl} 
\hline\hline             
ISO-L1551 & $F_{6.7}$~[mJy] & $\alpha_{6.7-14.3}$ 
& YSO type$^{d}$ & Status & Comments\\
\hline
  \multicolumn{6}{c}{\it New YSO candidates}\\ % To combine 6 columns into a single one
\hline
  1 & 1.56 $\pm$ 0.47 & --    & Class II$^{c}$ & New & Mid\\
  2 & 0.79:           & 0.97: & Class II ?     & New & \\
  3 & 4.95 $\pm$ 0.68 & 3.18  & Class II / III & New & \\
  5 & 1.44 $\pm$ 0.33 & 1.88  & Class II       & New & \\
\hline
  \multicolumn{6}{c}{\it Previously known YSOs} \\
\hline
  61 & 0.89 $\pm$ 0.58 & 1.77 & Class I & \object{HH 30} & Circumstellar disk\\
  96 & 38.34 $\pm$ 0.71 & 37.5& Class II& MHO 5          & Spectral type\\
\hline
\end{tabular}
\end{sidewaystable*}
%_____________________________________________________________
%                      A rotated One column Table in landscape  
%-------------------------------------------------------------
\begin{sidewaystable}
\caption{Summary for ISOCAM sources with mid-IR excess 
(YSO candidates).}\label{YSOtable}
\centering
\begin{tabular}{crrlcl} 
\hline\hline             
ISO-L1551 & $F_{6.7}$~[mJy] & $\alpha_{6.7-14.3}$ 
& YSO type$^{d}$ & Status & Comments\\
\hline
  \multicolumn{6}{c}{\it New YSO candidates}\\ % To combine 6 columns into a single one
\hline
  1 & 1.56 $\pm$ 0.47 & --    & Class II$^{c}$ & New & Mid\\
  2 & 0.79:           & 0.97: & Class II ?     & New & \\
  3 & 4.95 $\pm$ 0.68 & 3.18  & Class II / III & New & \\
  5 & 1.44 $\pm$ 0.33 & 1.88  & Class II       & New & \\
\hline
  \multicolumn{6}{c}{\it Previously known YSOs} \\
\hline
  61 & 0.89 $\pm$ 0.58 & 1.77 & Class I & \object{HH 30} & Circumstellar disk\\
  96 & 38.34 $\pm$ 0.71 & 37.5& Class II& MHO 5          & Spectral type\\
\hline
\end{tabular}
\end{sidewaystable}
%
%_____________________________________________________________
%                              Table longer than a single page  
%-------------------------------------------------------------
% All long tables will be placed automatically at the end, after 
%                                        \end{thebibliography}
%
\begin{longtab}
\begin{longtable}{lllrrr}
\caption{\label{kstars} Sample stars with absolute magnitude}\\
\hline\hline
Catalogue& $M_{V}$ & Spectral & Distance & Mode & Count Rate \\
\hline
\endfirsthead
\caption{continued.}\\
\hline\hline
Catalogue& $M_{V}$ & Spectral & Distance & Mode & Count Rate \\
\hline
\endhead
\hline
\endfoot
%%
Gl 33    & 6.37 & K2 V & 7.46 & S & 0.043170\\
Gl 66AB  & 6.26 & K2 V & 8.15 & S & 0.260478\\
Gl 68    & 5.87 & K1 V & 7.47 & P & 0.026610\\
         &      &      &      & H & 0.008686\\
Gl 86 
\footnote{Source not included in the HRI catalog. See Sect.~5.4.2 for details.}
         & 5.92 & K0 V & 10.91& S & 0.058230\\
\end{longtable}
\end{longtab}
%
%_____________________________________________________________
%                              Table longer than a single page
%                                             and in landscape 
%  In the preamble, use:       \usepackage{lscape}
%-------------------------------------------------------------
% All long tables will be placed automatically at the end, after
%                                        \end{thebibliography}
%
\begin{longtab}
\begin{landscape}
\begin{longtable}{lllrrr}
\caption{\label{kstars} Sample stars with absolute magnitude}\\
\hline\hline
Catalogue& $M_{V}$ & Spectral & Distance & Mode & Count Rate \\
\hline
\endfirsthead
\caption{continued.}\\
\hline\hline
Catalogue& $M_{V}$ & Spectral & Distance & Mode & Count Rate \\
\hline
\endhead
\hline
\endfoot
%%
Gl 33    & 6.37 & K2 V & 7.46 & S & 0.043170\\
Gl 66AB  & 6.26 & K2 V & 8.15 & S & 0.260478\\
Gl 68    & 5.87 & K1 V & 7.47 & P & 0.026610\\
         &      &      &      & H & 0.008686\\
Gl 86
\footnote{Source not included in the HRI catalog. See Sect.~5.4.2 for details.}
         & 5.92 & K0 V & 10.91& S & 0.058230\\
\end{longtable}
\end{landscape}
\end{longtab}
%
% Online Material
%_____________________________________________________________
%        Online appendices have to be placed at the end, after
%                                        \end{thebibliography}
%-------------------------------------------------------------
\end{thebibliography}

\Online

\begin{appendix} %First online appendix
\section{Background galaxy number counts and shear noise-levels}
Because the optical images used in this analysis...

\begin{figure*}
\centering
\includegraphics[width=16.4cm,clip]{1787f24.ps}
\caption{Plotted above...}
\label{appfig}
\end{figure*}

Because the optical images...
\end{appendix}

\begin{appendix} %Second online appendix
These studies, however, have faced...
\end{appendix}

\end{document}
%
%_____________________________________________________________
%        Some tables or figures are in the printed version and
%                      some are only in the electronic version
%-------------------------------------------------------------
%
% Leave all the tables or figures in the text, at their right place 
% and use the commands \onlfig{} and \onltab{}. These elements
% will be automatically placed at the end, in the section
% Online material.

\documentclass{aa}
...
\begin{document}
text of the paper...
\begin{figure*}%f1
\includegraphics[width=10.9cm]{1787f01.eps}
\caption{Shown in greyscale is a...}
\label{cl12301}}
\end{figure*}
...
from the intrinsic ellipticity distribution.
% Figure 2 available electronically only
\onlfig{
\begin{figure*}%f2
\includegraphics[width=11.6cm]{1787f02.eps}
\caption {Shown in greyscale...}
\label{cl1018}
\end{figure*}
}

% Figure 3 available electronically only
\onlfig{
\begin{figure*}%f3
\includegraphics[width=11.2cm]{1787f03.eps}
\caption{Shown in panels...}
\label{cl1059}
\end{figure*}
}

\begin{figure*}%f4
\includegraphics[width=10.9cm]{1787f04.eps}
\caption{Shown in greyscale is...}
\label{cl1232}}
\end{figure*}

\begin{table}%t1
\caption{Complexes characterisation.}\label{starbursts}
\centering
\begin{tabular}{lccc}
\hline \hline
Complex & $F_{60}$ & 8.6 &  No. of  \\
...
\hline
\end{tabular}
\end{table}
The second method produces...

% Figure 5 available electronically only
\onlfig{
\begin{figure*}%f5
\includegraphics[width=11.2cm]{1787f05.eps}
\caption{Shown in panels...}
\label{cl1238}}
\end{figure*}
}

As can be seen, in general the deeper...
% Table 2 available electronically only
\onltab{
\begin{table*}%t2
\caption{List of the LMC stellar complexes...}\label{Properties}
\centering
\begin{tabular}{lccccccccc}
\hline  \hline
Stellar & RA & Dec & ...
...
\hline
\end{tabular}
\end{table*}
}

% Table 3 available electronically only
\onltab{
\begin{table*}%t3
\caption{List of the derived...}\label{IrasFluxes}
\centering
\begin{tabular}{lcccccccccc}
\hline \hline
Stellar & $f12$ & $L12$ &...
...
\hline
\end{tabular}
\end{table*}
}
%
%-------------------------------------------------------------
%     For the online material, table longer than a single page
%                 In the preamble for landscape case, use : 
%                                          \usepackage{lscape}
%-------------------------------------------------------------
\documentclass{aa}
\usepackage[varg]{txfonts}
\usepackage{graphicx}
\usepackage{lscape}

\begin{document}
text of the paper
% Table will be print automatically at the end, in the section Online material.
\onllongtab{
\begin{longtable}{lrcrrrrrrrrl}
\caption{Line data and abundances ...}\\
\hline
\hline
Def & mol & Ion & $\lambda$ & $\chi$ & $\log gf$ & N & e &  rad & $\delta$ & $\delta$ 
red & References \\
\hline
\endfirsthead
\caption{Continued.} \\
\hline
Def & mol & Ion & $\lambda$ & $\chi$ & $\log gf$ & B & C &  rad & $\delta$ & $\delta$ 
red & References \\
\hline
\endhead
\hline
\endfoot
\hline
\endlastfoot
A & CH & 1 &3638 & 0.002 & $-$2.551 &  &  &  & $-$150 & 150 &  Jorgensen et al. (1996) \\                    
\end{longtable}
}% End onllongtab

% Or for landscape, large table:

\onllongtab{
\begin{landscape}
\begin{longtable}{lrcrrrrrrrrl}
...
\end{longtable}
\end{landscape}
}% End onllongtab

